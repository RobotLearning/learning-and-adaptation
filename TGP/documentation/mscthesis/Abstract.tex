\chapter*{Abstract}

Systems that work in a repetitive manner use Iterative Learning Control (ILC) algorithms to iteratively improve the performance over a given repeated task or trajectory. The feed-forward control signal is modified in each iteration to reduce the error or the deviation from the given reference trajectory. The limitation with ILC is that it assumes the task or the trajectory to be fixed over iterations. ILC cannot handle the cases when the trajectory is modified or changing over time, and the iterative learning controller must start learning from scratch. In this thesis, using CGP-UCB, an intuitive upper-confidence style algorithm based on Gaussian Processes (GP), it is shown that a significant amount of knowledge can be transferred even between cases where the reference trajectories are not the same. 

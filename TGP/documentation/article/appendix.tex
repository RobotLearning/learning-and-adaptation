\appendix

\section{Proof of Proposition \ref{Proposition2}}

To prove proposition \ref{Proposition2} we start with the following lemma:
\begin{lem}
$\forall \epsilon > 0 \ \exists \ \alpha_0(\epsilon, d) < \infty \ $ s.t. if $\omega_T = \mathcal{O}((\log T)^{d})$, then $\ \omega_{T} \leq \alpha_0 T^{\epsilon}, \ T \geq 1$.  \label{gamma-bound} 
\end{lem}
\begin{proof}
$\ \exists \ a < \infty \ $ s.t. $\ \omega_{T} \leq a (\log T)^{d}$. Take $\alpha_0 = a \max T^{-\epsilon} (\log T)^{d}$. Taking the derivative of $g(T) = T^{-\epsilon} (\log T)^{d}$:
\begin{align}
\frac{d(\log T)^{d-1} - \epsilon(\log T)^{d}}{T^{1+\epsilon}} = 0
\end{align}
The function $g(T)$ is continuous for $T \geq 1$ with $g(1) = 0$ and as $T \to \infty, \ g(T) \to 0$. Hence the only optimum attained in $(1, \infty)$ at $T_{opt} = \exp(d/\epsilon)$ with $g(T_{opt}) = \big(\frac{d}{e\epsilon}\big)^{d}$ is a global maximum. Taking $\alpha_0(\epsilon,d) = a\big(\frac{d}{e\epsilon}\big)^{d}$ then we can see that 
\begin{align}
\omega_{T} \leq a (\log T)^{d} \leq \alpha_0 T^{\epsilon}, \ T \geq 1. 
\end{align}
\end{proof}

Equipped with this lemma, we can prove Proposition \ref{Proposition2} for the squared exponential kernel~\footnote{The proof is the same for the linear kernel, but the exponent of the bound \eqref{T-bound} changes for Mat\'{e}rn kernels with $1 < \nu < \infty$.}:
%\begin{prop2}
%$\exists k$ s.t. $\tau(k) \geq N$ before $T \leq T_{max} = \big(\frac{N\bar{\alpha_0}}{\epsilon}\big)^{4}$ with high probability $p \geq 1 - \delta$.
%\end{prop2}
\begin{proof}
We can bound \eqref{pre-limit-T} using the previous lemma:
\begin{align}
\sum_{\episode=1}^{\numepisode(T)}\Lambda_\episode + \threshold \numepisode \leq R_{T} \leq \sqrt{\kappa T \beta_{T} \gamma_{T}} \leq \alpha T^{3/4} \label{T-bound}
\end{align}
where $\omega_T^2 = \kappa\beta_T\gamma_T = 2B\kappa\gamma_T + 300\kappa\gamma_T^2\log^{3}(T/\delta) = \mathcal{O}((\log T)^{2d+5})$ and $\alpha = \alpha_0(1/4,(2d+5)/2)$ for the squared exponential kernel. Since $\Lambda_\episode \geq 0, \ \episode = 1, \ldots, \numepisode$ we get w.h.p. $p \geq 1 - \delta$:
\begin{align}
\numepisode &\leq \frac{\alpha \ T^{3/4}}{\threshold} \\
\tau_m &= \frac{T}{\numepisode} \geq \frac{\threshold \ T^{1/4}}{\alpha}
\end{align}
where $\tau_m$ denotes the average of the stopping times $\tau(\episode), \episode = 1, \ldots, \numepisode$. But this means $\exists \episode \leq \numepisode$ s.t. $\tau(\episode) \geq \tau_m \geq (\threshold/\alpha)T^{1/4} \geq N$ before $T \leq T_{max} = \big(\frac{N\alpha}{\threshold}\big)^{4}$.
\end{proof}

\section{Numerical Examples}
%Table~\ref{table_parameters} shows the value of the constraints used throughout the numerical examples:

\begin{table}[h]
% increase table row spacing, adjust to taste
%\renewcommand{\arraystretch}{1.3}
\caption{Quadrocopter dynamical constraints}
\label{table_parameters}
%\vskip 0.15in
\begin{center}
\begin{small}
\begin{sc}
\begin{tabular}{cc}
\hline
\abovespace\belowspace
%Constraints & Trajectory Generation & Learning \\
Constraints & Values \\
\hline
\abovespace
%$f_{min}$ & $0.4\ m/s^2$ & $0.25\ m/s^2$ \\
$f_{min}$ & $0.25\ m/s^2$ \\
%$f_{max}$ & $4.5\ m/s^2$ & $5.5\ m/s^2$ \\
$f_{max}$ & $5.5\ m/s^2$ \\
%$\dot{f}_{max}$ & $27\ m/s^3$ & $51\ m/s^3$ \\
$\dot{f}_{max}$ & $51\ m/s^3$ \\
%$\dot{\phi}_{max}$ & $22\ rad/s$ & $25\ rad/s$ \\
$\dot{\phi}_{max}$ & $25\ rad/s$ \\
%$\ddot{\phi}_{max}$ & $150\ rad/s^2$ & $200\ rad/s^2$ \\
$\ddot{\phi}_{max}$ & $200\ rad/s^2$ \\
\hline
\end{tabular}
\end{sc}
\end{small}
\end{center}
\vskip -0.1in
\end{table}

\textit{Example 1}. As an example consider the effect of wind on the quadrotor operation. Assuming the wind, coming at an angle of $\theta$ from the horizontal axis, exerts a pressure $P_{wind}$ on the quadrotor with area $A$, the dynamics is modified as follows: 
\begin{equation}
\begin{aligned}
\ddot{y} &= -f_{\mathrm{coll}} \sin\phi + P_{wind} A sin(\theta + \phi) cos \theta \\
\ddot{z} &=  f_{\mathrm{coll}}\cos\phi - g + P_{wind} A sin(\theta + \phi) sin \theta \\
\dot{\phi} &= \omega_{x}
\end{aligned}
\label{windDisturbance}
\end{equation}
Mismatch in this case is only in the drift term.  Using squared Euclidean distance and forward Euler integration with time discretization $h$ the cost disturbance $\delta \cost_{t}(\sysInput;\state,\traj) = \cost_t - \hat{\cost}_t$ for the simple case of a perfectly horizontal wind, $\theta = 0$, can be calculated as follows:
\begin{equation}
\begin{aligned}
\delta \cost_{t}(\sysInput;\state,\traj) &= (h^{2}P_{wind}^{2}A^{2} - 2h^{2}P_{wind}A\ f_{\mathrm{coll}}) \sin^{2} \state(5) \\
&+ 2hP_{wind}A(\state(2) - \traj(2))\sin \state(5)
\end{aligned}
\label{QuadTheta0}
\end{equation}
In this case we effectively learn to compensate for this repeating disturbance $\delta \cost_{t}(\sysInput;\state,\traj): \mathbb{R}^{2} \times \mathbb{R} \times \mathbb{R} \mapsto \mathbb{R}$, as we do online $\alg$ optimization along the trajectory.

\textit{Example 2}. As another example consider mismatch in the quadrotor actuators. If the actual applied force is $f_{\mathrm{coll}}(1+a)$ for some small unknown $a$ the cost difference can be calculated as before:
\begin{equation}
\begin{aligned}
\delta \cost_{t}(\sysInput;\state,\traj) &= (h^{2}a^{2} + 2ha)f_{\mathrm{coll}}^{2} \\ 
&- 2h(\state(2) - \traj(2))a f_{\mathrm{coll}}\sin \state(5) \\
&+ 2h(\state(4) - \traj(4) - g)a f_{\mathrm{coll}} \cos \state(5)
\end{aligned}
\label{f_collMismatch}
\end{equation}


% This LaTeX was auto-generated from an M-file by MATLAB.
% To make changes, update the M-file and republish this document.



    
    

\subsection*{gTrig.m} 

\begin{par}
\textbf{Summary:} Compute moments of the saturating function $e*sin(x(i))$ and \$ e*cos(x(i))\$, where $x \sim\mathcal N(m,v)$ and $i$ is a (possibly empty) set of $I$ indices. The optional  scaling factor $e$ is a vector of length $I$. Optionally, compute derivatives of the moments.
\end{par} \vspace{1em}

\begin{verbatim}  [M, V, C, dMdm, dVdm, dCdm, dMdv, dVdv, dCdv] = gTrig(m, v, i, e)\end{verbatim}
    \begin{par}
\textbf{Input arguments:}
\end{par} \vspace{1em}
\begin{verbatim}m     mean vector of Gaussian                                    [ d       ]
v     covariance matrix                                          [ d  x  d ]
i     vector of indices of elements to augment                   [ I  x  1 ]
e     (optional) scale vector; default: 1                        [ I  x  1 ]\end{verbatim}
\begin{par}
\textbf{Output arguments:}
\end{par} \vspace{1em}
\begin{verbatim}M     output means                                              [ 2I       ]
V     output covariance matrix                                  [ 2I x  2I ]
C     inv(v) times input-output covariance                      [ d  x  2I ]
dMdm  derivatives of M w.r.t m                                  [ 2I x   d ]
dVdm  derivatives of V w.r.t m                                  [4II x   d ]
dCdm  derivatives of C w.r.t m                                  [2dI x   d ]
dMdv  derivatives of M w.r.t v                                  [ 2I x d^2 ]
dVdv  derivatives of V w.r.t v                                  [4II x d^2 ]
dCdv  derivatives of C w.r.t v                                  [2dI x d^2 ]\end{verbatim}
\begin{par}
Copyright (C) 2008-2013 by Marc Deisenroth, Andrew McHutchon, Joe Hall, and Carl Edward Rasmussen.
\end{par} \vspace{1em}
\begin{par}
Last modified: 2013-03-25
\end{par} \vspace{1em}

\begin{lstlisting}
function [M, V, C, dMdm, dVdm, dCdm, dMdv, dVdv, dCdv] = gTrig(m, v, i, e)
\end{lstlisting}


\subsection*{Code} 


\begin{lstlisting}
d = length(m); I = length(i); Ic = 2*(1:I); Is = Ic-1;
if nargin == 3, e = ones(I,1); else e = e(:); end; ee = reshape([e e]',2*I,1);
mi(1:I,1) = m(i); vi = v(i,i); vii(1:I,1) = diag(vi);     % short-hand notation

M(Is,1) = e.*exp(-vii/2).*sin(mi); M(Ic,1) = e.*exp(-vii/2).*cos(mi);    % mean

lq = -bsxfun(@plus,vii,vii')/2; q = exp(lq);
U1 = (exp(lq+vi)-q).*sin(bsxfun(@minus,mi,mi'));
U2 = (exp(lq-vi)-q).*sin(bsxfun(@plus,mi,mi'));
U3 = (exp(lq+vi)-q).*cos(bsxfun(@minus,mi,mi'));
U4 = (exp(lq-vi)-q).*cos(bsxfun(@plus,mi,mi'));
V(Is,Is) = U3 - U4; V(Ic,Ic) = U3 + U4; V(Is,Ic) = U1 + U2;
V(Ic,Is) = V(Is,Ic)'; V = ee*ee'.*V/2;                               % variance

C = zeros(d,2*I); C(i,Is) = diag(M(Ic)); C(i,Ic) = diag(-M(Is)); % inv(v) * cov

if nargout > 3                                           % compute derivatives?
  dVdm = zeros(2*I,2*I,d); dCdm = zeros(d,2*I,d); dVdv = zeros(2*I,2*I,d,d);
  dCdv = zeros(d,2*I,d,d); dMdm = C';
  for j = 1:I
    u = zeros(I,1); u(j) = 1/2;
    dVdm(Is,Is,i(j)) = e*e'.*(-U1.*bsxfun(@minus,u,u')+U2.*bsxfun(@plus,u,u'));
    dVdm(Ic,Ic,i(j)) = e*e'.*(-U1.*bsxfun(@minus,u,u')-U2.*bsxfun(@plus,u,u'));
    dVdm(Is,Ic,i(j)) = e*e'.*(U3.*bsxfun(@minus,u,u') +U4.*bsxfun(@plus,u,u'));
    dVdm(Ic,Is,i(j)) = dVdm(Is,Ic,i(j))';
    dVdv(Is(j),Is(j),i(j),i(j)) = exp(-vii(j)) * ...
                               (1+(2*exp(-vii(j))-1)*cos(2*mi(j)))*e(j)*e(j)/2;
    dVdv(Ic(j),Ic(j),i(j),i(j)) = exp(-vii(j)) * ...
                               (1-(2*exp(-vii(j))-1)*cos(2*mi(j)))*e(j)*e(j)/2;
    dVdv(Is(j),Ic(j),i(j),i(j)) = exp(-vii(j)) * ...
                                   (1-2*exp(-vii(j)))*sin(2*mi(j))*e(j)*e(j)/2;
    dVdv(Ic(j),Is(j),i(j),i(j)) = dVdv(Is(j),Ic(j),i(j),i(j));
    for k = [1:j-1 j+1:I]
      dVdv(Is(j),Is(k),i(j),i(k)) = (exp(lq(j,k)+vi(j,k)).*cos(mi(j)-mi(k)) ...
                         + exp(lq(j,k)-vi(j,k)).*cos(mi(j)+mi(k)))*e(j)*e(k)/2;
      dVdv(Is(j),Is(k),i(j),i(j)) = -V(Is(j),Is(k))/2;
      dVdv(Is(j),Is(k),i(k),i(k)) = -V(Is(j),Is(k))/2;
      dVdv(Ic(j),Ic(k),i(j),i(k)) = (exp(lq(j,k)+vi(j,k)).*cos(mi(j)-mi(k)) ...
                         - exp(lq(j,k)-vi(j,k)).*cos(mi(j)+mi(k)))*e(j)*e(k)/2;
      dVdv(Ic(j),Ic(k),i(j),i(j)) = -V(Ic(j),Ic(k))/2;
      dVdv(Ic(j),Ic(k),i(k),i(k)) = -V(Ic(j),Ic(k))/2;
      dVdv(Ic(j),Is(k),i(j),i(k)) = -(exp(lq(j,k)+vi(j,k)).*sin(mi(j)-mi(k)) ...
                         + exp(lq(j,k)-vi(j,k)).*sin(mi(j)+mi(k)))*e(j)*e(k)/2;
      dVdv(Ic(j),Is(k),i(j),i(j)) = -V(Ic(j),Is(k))/2;
      dVdv(Ic(j),Is(k),i(k),i(k)) = -V(Ic(j),Is(k))/2;
      dVdv(Is(j),Ic(k),i(j),i(k)) = (exp(lq(j,k)+vi(j,k)).*sin(mi(j)-mi(k)) ...
                         - exp(lq(j,k)-vi(j,k)).*sin(mi(j)+mi(k)))*e(j)*e(k)/2;
      dVdv(Is(j),Ic(k),i(j),i(j)) = -V(Is(j),Ic(k))/2;
      dVdv(Is(j),Ic(k),i(k),i(k)) = -V(Is(j),Ic(k))/2;
    end
    dCdm(i(j),Is(j),i(j)) = -M(Is(j)); dCdm(i(j),Ic(j),i(j)) = -M(Ic(j));
    dCdv(i(j),Is(j),i(j),i(j)) = -C(i(j),Is(j))/2;
    dCdv(i(j),Ic(j),i(j),i(j)) = -C(i(j),Ic(j))/2;
  end
  dMdv = permute(dCdm,[2 1 3])/2;

  dMdv = reshape(dMdv,[2*I d*d]);
  dVdv = reshape(dVdv,[4*I*I d*d]); dVdm = reshape(dVdm,[4*I*I d]);
  dCdv = reshape(dCdv,[d*2*I d*d]); dCdm = reshape(dCdm,[d*2*I d]);
end
\end{lstlisting}

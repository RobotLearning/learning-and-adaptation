
% This LaTeX was auto-generated from an M-file by MATLAB.
% To make changes, update the M-file and republish this document.



    
    
      \subsection{settings\_unicycle.m}

\begin{par}
\textbf{Summary:} Script to set up the unicycle scenario
\end{par} \vspace{1em}
\begin{par}
Copyright (C) 2008-2013 by Marc Deisenroth, Andrew McHutchon, Joe Hall, and Carl Edward Rasmussen.
\end{par} \vspace{1em}
\begin{par}
Last modified: 2013-04-02
\end{par} \vspace{1em}


\subsection*{High-Level Steps} 

\begin{enumerate}
\setlength{\itemsep}{-1ex}
   \item Define state and important indices
   \item Set up scenario
   \item Set up the plant structure
   \item Set up the policy structure
   \item Set up the cost structure
   \item Set up the GP dynamics model structure
   \item Parameters for policy optimization
   \item Plotting verbosity
   \item Some array initializations
\end{enumerate}


\subsection*{Code} 


\begin{lstlisting}
warning('off','all'); format short; format compact

% include some paths
try
  rd = '../../';
  addpath([rd 'base'],[rd 'util'],[rd 'gp'],[rd 'control'],[rd 'loss']);
catch
end

rand('seed',1); randn('seed',1);

% 1. Define state and important indices

% 1a. Full state representation (including all augmentations)
% - non-angle velocities
% - angular velocities
% - non-angles
% - angles
% - controls
%
%  1  dx      x velocity
%  2  dy      y velocity
%  3  dxc     x velocity of origin (unicycle coordinates)
%  4  dyc     y velocity of origin (unicycle coordinates)
%  5  dtheta  roll angular velocity
%  6  dphi    yaw angular velocity
%  7  dpsiw   wheel angular velocity
%  8  dpsif   pitch angular velocity
%  9  dpsit   turn table angular velocity
% 10  x       x position
% 11  y       y position
% 12  xc      x position of origin (unicycle coordinates)
% 13  yc      y position of origin (unicycle coordinates)
% 14  theta   roll angle
% 15  phi     yaw angle
% 16  psiw    wheel angle
% 17  psif    pitch angle
% 18  psit    turn table angle
% 19  ct      control torque for turn table
% 20  cw      control torque for wheel

% 1b. Important indices
% odei  indicies for the ode solver
% augi  indicies for variables augmented to the ode variables
% dyno  indicies for the output from the dynamics model and indicies to loss
% angi  indicies for variables treated as angles (using sin/cos representation)
% dyni  indicies for inputs to the dynamics model
% poli  indicies for variables that serve as inputs to the policy
% difi  indicies for training targets that are differences (rather than values)

odei = [5 6 7 8 9 10 11 14 15 16 17 18];
augi = [1 2 3 4 12 13];
dyno = [5 6 7 8 9 12 13 14 15 17];
angi = [];
dyni = [1 2 3 4 5 6 7 8 10];
poli = [1 2 3 4 5 6 7 8 9 10];
difi = [1 2 3 4 5 6 7 8 9 10];


% 2. Set up the scenario
dt = 0.15;                    % [s] sampling time
T = 10.0;                     % [s] initial prediction horizon time
H = ceil(T/dt);               % prediction steps (optimization horizon)
maxH = ceil(10.0/dt);         % max pred horizon
s = [0.02 0.02 0.02 0.02 0.02 0.1 0.1 0.02 0.02 0.02 0.02 0.02].^2;
S0 = diag(s);                 % initial state variance, 95% is +/- 11.4 degrees
mu0 = zeros(length(odei),1);  % initial state mean
N = 40;                       % number controller optimizations
J = 10;                       % initial J trajectories of length H
K = 1;                        % number of initial states for which we optimize

% 3. Set up the plant structure
plant.dynamics = @dynamics_unicycle; % dynamics ODE function
plant.augment = @augment_unicycle; % function to augment the state ODE variables
plant.constraint = inline('abs(x(8))>pi/2 | abs(x(11))>pi/2');  % ODE constraint
plant.noise = diag([0.01*ones(1,5) 0.003*ones(1,7)].^2);     % measurement noise
plant.dt = dt;
plant.ctrl = @zoh;                 % controller is zero order hold
plant.odei = odei;                 % indices to the varibles for the ODE solver
plant.augi = augi;                 % indices of augmented variables
plant.angi = angi;
plant.poli = poli;
plant.dyno = dyno;
plant.dyni = dyni;
plant.difi = difi;
plant.prop = @propagated;

% 4. Set up the policy structure
policy.fcn = @(policy,m,s)conCat(@conlin,@gSat,policy,m,s); % linear policy
policy.maxU = [10 50];                                      % max. amplitude of
policy.p.w = 1e-2*randn(length(policy.maxU),length(poli));  % weight matrix
policy.p.b = zeros(length(policy.maxU),1);                  % bias
                                                            % torques

% 5. Set up the cost structure
cost.fcn = @loss_unicycle;                  % cost function
cost.gamma = 1;                             % discount factor
cost.p = [0.22 0.81];                       % radius of wheel and length of rod
cost.width = 1;                           % cost function width
cost.expl = 0;                              % exploration parameter (UCB)

% 6. Set up the GP dynamics model structure
dynmodel.fcn = @gp1d;                % function for GP predictions
dynmodel.train = @train;             % function to train dynamics model
dynmodel.induce = zeros(300,0,1);    % shared inducing inputs (sparse GP)



% 7. Parameters for policy optimization
opt.length = -150;                   % max. number of function evaluations
opt.MFEPLS = 20;                     % max. number of function evaluations
                                     % per line search
opt.verbosity = 2;                   % verbosity: specifies how much
                                     % information is displayed during
                                     % policy learning. Options: 0-3
opt.method = 'BFGS';                 % optimization algorithm
trainOpt = [300 300];                % defines the max. number of line searches
                                     % when training the GP dynamics models
                                     % trainOpt(1): full GP,
                                     % trainOpt(2): sparse GP (FITC)


% 8. Plotting verbosity
plotting.verbosity = 2;              % 0: no plots
                                     % 1: some plots
                                     % 2: all plots

% 9. Some initializations
x = []; y = [];
fantasy.mean = cell(1,N); fantasy.std = cell(1,N);
realCost = cell(1,N); M = cell(N,1); Sigma = cell(N,1);
\end{lstlisting}

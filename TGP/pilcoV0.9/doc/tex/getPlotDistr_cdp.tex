
% This LaTeX was auto-generated from an M-file by MATLAB.
% To make changes, update the M-file and republish this document.



    
    
      \subsection{getPlotDistr\_cdp.m}

\begin{par}
\textbf{Summary:} Compute means and covariances of the Cartesian coordinates of the tips both the inner and outer pendulum assuming that the joint state $x$ of the cart-double-pendulum system is Gaussian, i.e., $x\sim N(m, s)$
\end{par} \vspace{1em}

\begin{verbatim}   function [M1, S1, M2, S2] = getPlotDistr_cdp(m, s, ell1, ell2)\end{verbatim}
    \begin{par}
\textbf{Input arguments:}
\end{par} \vspace{1em}
\begin{verbatim}m       mean of full state                                    [6 x 1]
s       covariance of full state                              [6 x 6]
ell1    length of inner pendulum
ell2    length of outer pendulum\end{verbatim}
\begin{verbatim}Note: this code assumes that the following order of the state:
       1: cart pos.,
       2: cart vel.,
       3: pend1 angular velocity,
       4: pend2 angular velocity,
       5: pend1 angle,
       6: pend2 angle\end{verbatim}
\begin{par}
\textbf{Output arguments:}
\end{par} \vspace{1em}
\begin{verbatim}M1      mean of tip of inner pendulum                         [2 x 1]
S1      covariance of tip of inner pendulum                   [2 x 2]
M2      mean of tip of outer pendulum                         [2 x 1]
S2      covariance of tip of outer pendulum                   [2 x 2]\end{verbatim}
\begin{par}
Copyright (C) 2008-2013 by Marc Deisenroth, Andrew McHutchon, Joe Hall, and Carl Edward Rasmussen.
\end{par} \vspace{1em}
\begin{par}
Last modification: 2013-03-06
\end{par} \vspace{1em}


\subsection*{High-Level Steps} 

\begin{enumerate}
\setlength{\itemsep}{-1ex}
   \item Augment input distribution to complex angle representation
   \item Compute means of tips of pendulums (in Cartesian coordinates)
   \item Compute covariances of tips of pendulums (in Cartesian coordinates)
\end{enumerate}

\begin{lstlisting}
function [M1, S1, M2, S2] = getPlotDistr_cdp(m, s, ell1, ell2)
\end{lstlisting}


\subsection*{Code} 


\begin{lstlisting}
% 1. Augment input distribution (complex representation)
[m1 s1 c1] = gTrig(m, s, [5 6], [ell1, ell2]); % map input through sin/cos
m1 = [m; m1];        % mean of joint
c1 = s*c1;           % cross-covariance between input and prediction
s1 = [s c1; c1' s1]; % covariance of joint

% 2. Mean of the tips of the pendulums (Cart. coord.)
M1 = [m1(1) - m1(7); m1(8)];     % p2: E[x -l1\sin\theta_2]; E[l2\cos\theta_2]
M2 = [M1(1) - m1(9); M1(2) + m1(10)];               % p3: mean of cart. coord.

% 2. Put covariance matrices together (Cart. coord.)
% first set of coordinates (tip of 1st pendulum)
S1(1,1) = s1(1,1) + s1(7,7) -2*s1(1,7);
S1(2,2) = s1(8,8);
S1(1,2) = s1(1,8) - s1(7,8);
S1(2,1) = S1(1,2)';

% second set of coordinates (tip of 2nd pendulum)
S2(1,1) = S1(1,1) + s1(9,9)  + 2*(s1(1,9) - s1(7,9));
S2(2,2) = s1(8,8) + s1(10,10) + 2*s1(8,10);
S2(1,2) = s1(1,8) - s1(7,8) - s1(9,8) ...
        + s1(1,10) - s1(7,10) - s1(9,10);
S2(2,1) = S2(1,2)';

% make sure we have proper covariances (sometimes numerical problems occur)
try
  chol(S1);
catch
  warning('matrix S1 not pos.def. (getPlotDistr)');
  S1 = S1 + (1e-6 - min(eig(S1)))*eye(2);
end

try
  chol(S2);
catch
  warning('matrix S2 not pos.def. (getPlotDistr)');
  S2 = S2 + (1e-6 - min(eig(S2)))*eye(2);
end
\end{lstlisting}

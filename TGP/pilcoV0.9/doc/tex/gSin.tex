
% This LaTeX was auto-generated from an M-file by MATLAB.
% To make changes, update the M-file and republish this document.



    
    

\subsection*{gSin.m} 

\begin{par}
\textbf{Summary:} Compute moments of the saturating function $e*sin(x(i))$, where $x \sim\mathcal N(m,v)$ and $i$ is a (possibly empty) set of $I$ indices. The optional  scaling factor $e$ is a vector of length $I$. Optionally, compute derivatives of the moments.
\end{par} \vspace{1em}

\begin{verbatim}  function [M, V, C, dMdm, dVdm, dCdm, dMdv, dVdv, dCdv] = gSin(m, v, i, e)\end{verbatim}
    \begin{par}
\textbf{Input arguments:}
\end{par} \vspace{1em}
\begin{verbatim}m     mean vector of Gaussian                                    [ d       ]
v     covariance matrix                                          [ d  x  d ]
i     vector of indices of elements to augment                   [ I  x  1 ]
e     (optional) scale vector; default: 1                        [ I  x  1 ]\end{verbatim}
\begin{par}
\textbf{Output arguments:}
\end{par} \vspace{1em}
\begin{verbatim}M     output means                                               [ I       ]
V     output covariance matrix                                   [ I  x  I ]
C     inv(v) times input-output covariance                       [ d  x  I ]
dMdm  derivatives of M w.r.t m                                   [ I  x  d ]
dVdm  derivatives of V w.r.t m                                   [I^2 x  d ]
dCdm  derivatives of C w.r.t m                                   [d*I x  d ]
dMdv  derivatives of M w.r.t v                                   [ I  x d^2]
dVdv  derivatives of V w.r.t v                                   [I^2 x d^2]
dCdv  derivatives of C w.r.t v                                   [d*I x d^2]\end{verbatim}
\begin{par}
Copyright (C) 2008-2013 by Marc Deisenroth, Andrew McHutchon, Joe Hall, and Carl Edward Rasmussen.
\end{par} \vspace{1em}
\begin{par}
Last modified: 2013-03-25
\end{par} \vspace{1em}

\begin{lstlisting}
function [M, V, C, dMdm, dVdm, dCdm, dMdv, dVdv, dCdv] = gSin(m, v, i, e)
\end{lstlisting}


\subsection*{Code} 


\begin{lstlisting}
d = length(m); I = length(i);
if nargin == 3, e = ones(I,1); else e = e(:); end          % unit column default
mi(1:I,1) = m(i); vi = v(i,i); vii(1:I,1) = diag(vi);      % short-hand notation

M = e.*exp(-vii/2).*sin(mi);                                              % mean

lq = -bsxfun(@plus,vii,vii')/2; q = exp(lq);
V = (exp(lq+vi)-q).*cos(bsxfun(@minus,mi,mi')) - ...
                                      (exp(lq-vi)-q).*cos(bsxfun(@plus,mi,mi'));
V = e*e'.*V/2;                                                        % variance

C = zeros(d,I); C(i,:) = diag(e.*exp(-vii/2).*cos(mi));       % inv(v) times cov

if nargout > 3                                            % compute derivatives?
  dVdm = zeros(I,I,d); dCdm = zeros(d,I,d); dVdv = zeros(I,I,d,d);
  dCdv = zeros(d,I,d,d); dMdm = C';
  U1 = -(exp(lq+vi)-q).*sin(bsxfun(@minus,mi,mi'));
  U2 = (exp(lq-vi)-q).*sin(bsxfun(@plus,mi,mi'));
  for j = 1:I
    u = zeros(I,1); u(j) = 1/2;
    dVdm(:,:,i(j)) = e*e'.*(U1.*bsxfun(@minus,u,u') + U2.*bsxfun(@plus,u,u'));
    dVdv(j,j,i(j),i(j)) = exp(-vii(j)) * ...
                                (1+(2*exp(-vii(j))-1)*cos(2*mi(j)))*e(j)*e(j)/2;
    for k = [1:j-1 j+1:I]
      dVdv(j,k,i(j),i(k)) = (exp(lq(j,k)+vi(j,k)).*cos(mi(j)-mi(k)) + ...
                            exp(lq(j,k)-vi(j,k)).*cos(mi(j)+mi(k)))*e(j)*e(k)/2;
      dVdv(j,k,i(j),i(j)) = -V(j,k)/2;
      dVdv(j,k,i(k),i(k)) = -V(j,k)/2;
    end
    dCdm(i(j),j,i(j)) = -M(j);
    dCdv(i(j),j,i(j),i(j)) = -C(i(j),j)/2;
  end
  dMdv = permute(dCdm,[2 1 3])/2;

  dMdv = reshape(dMdv,[I d*d]);
  dVdv = reshape(dVdv,[I*I d*d]); dVdm = reshape(dVdm,[I*I d]);
  dCdv = reshape(dCdv,[d*I d*d]); dCdm = reshape(dCdm,[d*I d]);
end
\end{lstlisting}

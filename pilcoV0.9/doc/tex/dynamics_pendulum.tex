
% This LaTeX was auto-generated from an M-file by MATLAB.
% To make changes, update the M-file and republish this document.



    
    

\subsection*{dynamics\_pendulum.m} 

\begin{par}
\textbf{Summary:} Implements ths ODE for simulating the pendulum dynamics, where an input torque f can be applied
\end{par} \vspace{1em}

\begin{verbatim}  function dz = dynamics_pendulum(t,z,u)\end{verbatim}
    \begin{par}
\textbf{Input arguments:}
\end{par} \vspace{1em}

\begin{lstlisting}
%		t     current time step (called from ODE solver)
%   z     state                                                    [2 x 1]
%   u     (optional): torque f(t) applied to pendulum
%
% *Output arguments:*
%
%   dz    if 3 input arguments:      state derivative wrt time
%
%   Note: It is assumed that the state variables are of the following order:
%         dtheta:  [rad/s] angular velocity of pendulum
%         theta:   [rad]   angle of pendulum
%
% A detailed derivation of the dynamics can be found in:
%
% M.P. Deisenroth:
% Efficient Reinforcement Learning Using Gaussian Processes, Appendix C,
% KIT Scientific Publishing, 2010.
%
%
% Copyright (C) 2008-2013 by
% Marc Deisenroth, Andrew McHutchon, Joe Hall, and Carl Edward Rasmussen.
%
% Last modified: 2013-03-18

function dz = dynamics_pendulum(t,z,u)
\end{lstlisting}


\subsection*{Code} 


\begin{lstlisting}
l = 1;    % [m]        length of pendulum
m = 1;    % [kg]       mass of pendulum
g = 9.82; % [m/s^2]    acceleration of gravity
b = 0.01; % [s*Nm/rad] friction coefficient

dz = zeros(2,1);
dz(1) = ( u(t) - b*z(1) - m*g*l*sin(z(2))/2 ) / (m*l^2/3);
dz(2) = z(1);
\end{lstlisting}

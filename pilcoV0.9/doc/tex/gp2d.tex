
% This LaTeX was auto-generated from an M-file by MATLAB.
% To make changes, update the M-file and republish this document.



    
    
      \subsection{gp2d.m}

\begin{par}
\textbf{Summary:} Compute joint predictions and derivatives for multiple GPs with uncertain inputs. Does not consider the uncertainty about the underlying function (in prediction), hence, only the GP mean function is considered. Therefore, this representation is equivalent to a regularized RBF network. If gpmodel.nigp exists, individial noise contributions are added.
\end{par} \vspace{1em}
\begin{verbatim}function [M, S, V, dMdm, dSdm, dVdm, dMds, dSds, dVds, dMdi, dSdi, dVdi, ...
                   dMdt, dSdt, dVdt, dMdX, dSdX, dVdX] = gp2d(gpmodel, m, s)\end{verbatim}
\begin{par}
\textbf{Input arguments:}
\end{par} \vspace{1em}
\begin{verbatim}gpmodel    GP model struct
  hyp      log-hyper-parameters                                  [D+2 x  E ]
  inputs   training inputs                                       [ n  x  D ]
  targets  training targets                                      [ n  x  E ]
  nigp     (optional) individual noise variance terms            [ n  x  E ]
m          mean of the test distribution                         [ D  x  1 ]
s          covariance matrix of the test distribution            [ D  x  D ]\end{verbatim}
\begin{par}
\textbf{Output arguments:}
\end{par} \vspace{1em}
\begin{verbatim}M          mean of pred. distribution                            [ E  x  1 ]
S          covariance of the pred. distribution                  [ E  x  E ]
V          inv(s) times covariance between input and output      [ D  x  E ]
dMdm       output mean by input mean                             [ E  x  D ]
dSdm       output covariance by input mean                       [E*E x  D ]
dVdm       inv(s)*input-output covariance by input mean          [D*E x  D ]
dMds       ouput mean by input covariance                        [ E  x D*D]
dSds       output covariance by input covariance                 [E*E x D*D]
dVds       inv(s)*input-output covariance by input covariance    [D*E x D*D]\end{verbatim}
\begin{verbatim}dMdi      output mean by inputs                                  [ E  x n*D]
dSdi      output covariance by inputs                            [E*E x n*D]
dVdi      inv(s) times input-output covariance by inputs         [D*E x n*D]
dMdt      output mean by targets                                 [ E  x n*E]
dSdt      output covariance by targets                           [E*E x n*E]
dVdt      inv(s) times input-output covariance by targets        [D*E x n*E]
dMdX      output mean by hyperparameters                         [ E  x P*E]
dSdX      output covariance by hyperparameters                   [E*E x P*E]
dVdX      inv(s) times input-output covariance by hyper-par.     [D*E x P*E]\end{verbatim}
\begin{par}
Copyright (C) 2008-2013 by Marc Deisenroth, Andrew McHutchon, Joe Hall, and Carl Edward Rasmussen.
\end{par} \vspace{1em}
\begin{par}
Last modified: 2013-05-24
\end{par} \vspace{1em}


\subsection*{High-Level Steps} 

\begin{enumerate}
\setlength{\itemsep}{-1ex}
   \item If necessary, re-compute cached variables
   \item Compute predicted mean and inv(s) times input-output covariance
   \item Compute predictive covariance matrix, non-central moments
   \item Centralize moments
   \item Vectorize derivatives
\end{enumerate}

\begin{lstlisting}
function [M, S, V, dMdm, dSdm, dVdm, dMds, dSds, dVds, dMdi, dSdi, dVdi, ...
  dMdt, dSdt, dVdt, dMdX, dSdX, dVdX] = gp2d(gpmodel, m, s)
\end{lstlisting}


\subsection*{Code} 


\begin{lstlisting}
input = gpmodel.inputs;  target = gpmodel.targets; X = gpmodel.hyp;

if nargout < 4; [M, S, V] = gp2(gpmodel, m, s); return; end

persistent K iK oldX oldIn oldOut beta oldn;           % cache some variables
D = size(input,2);         % number of examples and dimension of input space
[n, E] = size(target);               % number of examples and number of outputs
X = reshape(X, D+2, E);

% 1) If necessary, re-compute cached variables
if numel(X) ~= numel(oldX) || isempty(iK) ||  n ~= oldn || ...
    sum(any(X ~= oldX)) || sum(any(oldIn ~= input)) || ...
    sum(any(oldOut ~= target))
  oldX = X; oldIn = input; oldOut = target; oldn = n;
  K = zeros(n,n,E); iK = K; beta = zeros(n,E);

  % compute K and inv(K) and beta
  for i=1:E
    inp = bsxfun(@rdivide,input,exp(X(1:D,i)'));
    K(:,:,i) = exp(2*X(D+1,i)-maha(inp,inp)/2);
    if isfield(gpmodel,'nigp')
      L = chol(K(:,:,i) + exp(2*X(D+2,i))*eye(n) + diag(gpmodel.nigp(:,i)))';
    else
      L = chol(K(:,:,i) + exp(2*X(D+2,i))*eye(n))';
    end
    iK(:,:,i) = L'\(L\eye(n));
    beta(:,i) = L'\(L\gpmodel.targets(:,i));
  end

end

% initializations
k = zeros(n,E); M = zeros(E,1); V = zeros(D,E); S = zeros(E);
dMds = zeros(E,D,D); dSdm = zeros(E,E,D); r = zeros(1,D);
dSds = zeros(E,E,D,D); dVds = zeros(D,E,D,D); T = zeros(D);
tlbdi = zeros(n,D); dMdi = zeros(E,n,D); dMdt = zeros(E,n,E);
dVdt = zeros(D,E,n,E); dVdi = zeros(D,E,n,D); dSdt = zeros(E,E,n,E);
dSdi = zeros(E,E,n,D); dMdX = zeros(E,D+2,E); dSdX = zeros(E,E,D+2,E);
dVdX = zeros(D,E,D+2,E); Z = zeros(n,D);
bdX = zeros(n,E,D); kdX = zeros(n,E,D+1);

% centralize training inputs
inp = bsxfun(@minus,input,m');

% 2) compute predicted mean and input-output covariance
for i = 1:E
  % first some useful intermediate terms
  K2 = K(:,:,i)+exp(2*X(D+2,i))*eye(n);                         % K + sigma^2*I
  inp2 = bsxfun(@rdivide,input,exp(X(1:D,i)'));
  ii = bsxfun(@rdivide,input,exp(2*X(1:D,i)'));
  R = s+diag(exp(2*X(1:D,i)));
  L = diag(exp(-X(1:D,i)));
  B = L*s*L+eye(D); iR = L/B*L;
  t = inp*iR;
  l = exp(-sum(t.*inp,2)/2); lb = l.*beta(:,i);
  tliK = t'*bsxfun(@times,l,iK(:,:,i));
  liK = K2\l;
  tlb = bsxfun(@times,t,lb);

  c = exp(2*X(D+1,i))/sqrt(det(R))*exp(sum(X(1:D,i)));
  detdX = diag(bsxfun(@times,det(R)*iR',2.*exp(2.*X(1:D,i))));    % d(det R)/dX
  cdX = -0.5*c/det(R).*detdX'+ c.*ones(1,D);       % derivs w.r.t length-scales
  dldX = bsxfun(@times,l,bsxfun(@times,t,2.*exp(2*X(1:D,i)')).*t./2);

  M(i) = sum(lb)*c;                                            % predicted mean

  iK2beta = K2\beta(:,i);
  dMds(i,:,:) = c*t'*tlb/2-iR*M(i)/2;
  dMdX(i,D+2,i) = -c*sum(l.*(2*exp(2*X(D+2,i))*iK2beta));           % OK
  dMdX(i,D+1,i) = -dMdX(i,(i-1)*(D+2)+D+2);

  dVdX(:,i,D+2,i) = -((l.*(2*exp(2*X(D+2,i))*iK2beta))'*t*c)';
  dVdX(:,i,D+1,i) = -dVdX(:,i,D+2,i);

  dsi = -bsxfun(@times,inp2,2.*inp2);                 % d(sum(inp2.*inp2,2))/dX
  dslb = zeros(1,D);

  for d = 1:D
    sqdi = K(:,:,i).*bsxfun(@minus,ii(:,d),ii(:,d)');
    sqdiBi = sqdi*beta(:,i);
    tlbdi(:,d) = sqdi*liK.*beta(:,i) + sqdiBi.*liK;
    tlbdi2 = -tliK*(-bsxfun(@times,sqdi,beta(:,i))'-diag(sqdiBi));
    dVdi(:,i,:,d) = c*(iR(:,d)*lb' - bsxfun(@times,t,tlb(:,d))' + tlbdi2);
    dsqdX = bsxfun(@plus,dsi(:,d),dsi(:,d)') + 4.*inp2(:,d)*inp2(:,d)';
    dKdX = -K(:,:,i).*dsqdX./2;                               % dK/dX(1:D)
    dKdXbeta = dKdX*beta(:,i);
    bdX(:,i,d) = -K2\dKdXbeta;                        % dbeta/dX
    dslb(d) = -liK'*dKdXbeta + beta(:,i)'*dldX(:,d);
    dlb = dldX(:,d).*beta(:,i) + l.*bdX(:,i,d);
    dtdX = inp*(-bsxfun(@times,iR(:,d),2.*exp(2*X(d,i))*iR(d,:)));
    dlbt = lb'*dtdX + dlb'*t;
    dVdX(:,i,d,i) = (dlbt'*c + cdX(d)*(lb'*t)');
  end % d

  dMdi(i,:,:) = c*(tlbdi - tlb);
  dMdt(i,:,i) = c*liK';
  dMdX(i,1:D,i) = cdX.*sum(beta(:,i).*l) + c.*dslb;
  v = bsxfun(@rdivide,inp,exp(X(1:D,i)'));
  k(:,i) = 2*X(D+1,i)-sum(v.*v,2)/2;
  V(:,i) = t'*lb*c;                                   % input-output covariance

  for d = 1:D
    dVds(d,i,:,:) = c*bsxfun(@times,t,t(:,d))'*tlb/2 - iR*V(d,i)/2 ...
      - V(:,i)*iR(d,:)/2 -iR(:,d)*V(:,i)'/2;
    kdX(:,i,d) = bsxfun(@times,v(:,d),v(:,d));
  end % d

  dVdt(:,i,:,i) = c*tliK;
  kdX(:,i,D+1) = 2*ones(1,n);                       % pre-computation for later

end % i
dMdm = V';                                                % derivatives w.r.t m
dVdm = 2*permute(dMds,[2 1 3]);


% 3) predictive covariance matrix (non-central moments)
for i = 1:E
  K2 = K(:,:,i)+exp(2*X(D+2,i))*eye(n);
  ii = bsxfun(@rdivide,inp,exp(2*X(1:D,i)'));

  for j = 1:i % if i==j: diagonal elements of S; see Marc's thesis around eq. (2.26)
    R = s*diag(exp(-2*X(1:D,i))+exp(-2*X(1:D,j)))+eye(D); t = 1./sqrt(det(R));
    if rcond(R) < 1e-15; fprintf('R-matrix in gp2d ill-conditioned'); keyboard; end
    iR = R\eye(D);
    ij = bsxfun(@rdivide,inp,exp(2*X(1:D,j)'));
    L = exp(bsxfun(@plus,k(:,i),k(:,j)')+maha(ii,-ij,R\s/2)); % called Q in thesis
    A = beta(:,i)*beta(:,j)'; A = A.*L; ssA = sum(sum(A));
    S(i,j) = t*ssA; S(j,i) = S(i,j);

    zzi = ii*(R\s);
    zzj = ij*(R\s);
    zi = ii/R; zj = ij/R;

    tdX  = -0.5*t*sum(iR'.*bsxfun(@times,s,-2*exp(-2*X(1:D,i)')-2*exp(-2*X(1:D,i)')));
    tdXi = -0.5*t*sum(iR'.*bsxfun(@times,s,-2*exp(-2*X(1:D,i)')));
    tdXj = -0.5*t*sum(iR'.*bsxfun(@times,s,-2*exp(-2*X(1:D,j)')));
    bLiKi = iK(:,:,j)*(L'*beta(:,i)); bLiKj = iK(:,:,i)*(L*beta(:,j));

    Q2 = R\s/2;
    aQ = ii*Q2; bQ = ij*Q2;

    for d = 1:D

      Z(:,d) = exp(-2*X(d,i))*(A*zzj(:,d) + sum(A,2).*(zzi(:,d) - inp(:,d)))...
        + exp(-2*X(d,j))*((zzi(:,d))'*A + sum(A,1).*(zzj(:,d) - inp(:,d))')';
      Q = bsxfun(@minus,inp(:,d),inp(:,d)');
      B = K(:,:,i).*Q;
      Z(:,d) = Z(:,d)+exp(-2*X(d,i))*(B*beta(:,i).*bLiKj+beta(:,i).*(B*bLiKj));

      if i~=j; B = K(:,:,j).*Q; end

      Z(:,d) = Z(:,d)+exp(-2*X(d,j))*(bLiKi.*(B*beta(:,j))+B*bLiKi.*beta(:,j));
      B = bsxfun(@plus,zi(:,d),zj(:,d)').*A;
      r(d) = sum(sum(B))*t;
      T(d,1:d) = sum(zi(:,1:d)'*B,2) + sum(B*zj(:,1:d))';
      T(1:d,d) = T(d,1:d)';

      if i==j
        RTi =  bsxfun(@times,s,(-2*exp(-2*X(1:D,i)')-2*exp(-2*X(1:D,j)')));
        diRi = -R\bsxfun(@times,RTi(:,d),iR(d,:));
      else
        RTi = bsxfun(@times,s,-2*exp(-2*X(1:D,i)'));
        RTj = bsxfun(@times,s,-2*exp(-2*X(1:D,j)'));
        diRi = -R\bsxfun(@times,RTi(:,d),iR(d,:));
        diRj = -R\bsxfun(@times,RTj(:,d),iR(d,:));
        QdXj = diRj*s/2; % dQ2/dXj
      end

      QdXi = diRi*s/2; % dQ2/dXj

      if i==j
        daQi = ii*QdXi + bsxfun(@times,-2*ii(:,d),Q2(d,:)); % d(ii*Q)/dXi
        dsaQi = sum(daQi.*ii,2) - 2.*aQ(:,d).*ii(:,d); dsaQj = dsaQi;
        dsbQi = dsaQi; dsbQj = dsbQi;
        dm2i = -2*daQi*ii' + 2*(bsxfun(@times,aQ(:,d),ii(:,d)')...
          +bsxfun(@times,ii(:,d),aQ(:,d)')); dm2j = dm2i; % -2*aQ*ij'/di
      else
        dbQi = ij*QdXi;  % d(ij*Q)/dXi
        dbQj = ij*QdXj + bsxfun(@times,-2*ij(:,d),Q2(d,:)); % d(ij*Q)/dXj
        daQi = ii*QdXi + bsxfun(@times,-2*ii(:,d),Q2(d,:)); % d(ii*Q)/dXi
        daQj = ii*QdXj; % d(ii*Q)/dXj

        dsaQi = sum(daQi.*ii,2) - 2.*aQ(:,d).*ii(:,d);
        dsaQj = sum(daQj.*ii,2);
        dsbQi = sum(dbQi.*ij,2);
        dsbQj = sum(dbQj.*ij,2) - 2.*bQ(:,d).*ij(:,d);
        dm2i = -2*daQi*ij'; % second part of the maha(..) function wrt Xi
        dm2j = -2*ii*(dbQj)'; % second part of the maha(..) function wrt Xj
      end

      dm1i = bsxfun(@plus,dsaQi,dsbQi'); % first part of the maha(..) function wrt Xi
      dm1j = bsxfun(@plus,dsaQj,dsbQj'); % first part of the maha(..) function wrt Xj
      dmahai = dm1i-dm2i;
      dmahaj = dm1j-dm2j;

      if i==j
        LdXi = L.*(dmahai + bsxfun(@plus,kdX(:,i,d),kdX(:,j,d)'));
        dSdX(i,i,d,i) = beta(:,i)'*LdXi*beta(:,j);
      else
        LdXi = L.*(dmahai + bsxfun(@plus,kdX(:,i,d),zeros(n,1)'));
        LdXj = L.*(dmahaj + bsxfun(@plus,zeros(n,1),kdX(:,j,d)'));
        dSdX(i,j,d,i) = beta(:,i)'*LdXi*beta(:,j);
        dSdX(i,j,d,j) = beta(:,i)'*LdXj*beta(:,j);
      end

    end % d

    if i==j
      dSdX(i,i,1:D,i) = reshape(dSdX(i,i,1:D,i),D,1) + reshape(bdX(:,i,:),n,D)'*(L+L')*beta(:,i);
      dSdX(i,i,1:D,i) = reshape(t*dSdX(i,i,1:D,i),D,1)' + tdX*ssA;
      dSdX(i,i,D+2,i) = 2*exp(2*X(D+2,i))*t*(-sum(beta(:,i).*bLiKi)-sum(beta(:,i).*bLiKi));
    else
      dSdX(i,j,1:D,i) = reshape(dSdX(i,j,1:D,i),D,1) + reshape(bdX(:,i,:),n,D)'*(L*beta(:,j));
      dSdX(i,j,1:D,j) = reshape(dSdX(i,j,1:D,j),D,1) + reshape(bdX(:,j,:),n,D)'*(L'*beta(:,i));
      dSdX(i,j,1:D,i) = reshape(t*dSdX(i,j,1:D,i),D,1)' + tdXi*ssA;
      dSdX(i,j,1:D,j) = reshape(t*dSdX(i,j,1:D,j),D,1)' + tdXj*ssA;
      dSdX(i,j,D+2,i) = 2*exp(2*X(D+2,i))*t*(-beta(:,i)'*bLiKj);
      dSdX(i,j,D+2,j) = 2*exp(2*X(D+2,j))*t*(-beta(:,j)'*bLiKi);
    end

    dSdm(i,j,:) = r - M(i)*dMdm(j,:)-M(j)*dMdm(i,:); dSdm(j,i,:) = dSdm(i,j,:);
    T = (t*T-S(i,j)*diag(exp(-2*X(1:D,i))+exp(-2*X(1:D,j)))/R)/2;
    T = T - reshape(M(i)*dMds(j,:,:) + M(j)*dMds(i,:,:),D,D);
    dSds(i,j,:,:) = T; dSds(j,i,:,:) = T;

    if i==j
      dSdt(i,i,:,i) = (beta(:,i)'*(L+L'))/(K2)*t ...
        - 2*dMdt(i,:,i)*M(i);
      dSdX(i,j,:,i) = reshape(dSdX(i,j,:,i),1,D+2) - M(i)*dMdX(j,:,j)-M(j)*dMdX(i,:,i);
    else
      dSdt(i,j,:,i) = (beta(:,j)'*L')/(K2)*t ...
        - dMdt(i,:,i)*M(j);
      dSdt(i,j,:,j) = beta(:,i)'*L/(K(:,:,j)+exp(2*X(D+2,j))*eye(n))*t ...
        - dMdt(j,:,j)*M(i);
      dSdt(j,i,:,:) = dSdt(i,j,:,:);
      dSdX(i,j,:,j) = reshape(dSdX(i,j,:,j),1,D+2) - M(i)*dMdX(j,:,j);
      dSdX(i,j,:,i) = reshape(dSdX(i,j,:,i),1,D+2) - M(j)*dMdX(i,:,i);
    end

    dSdi(i,j,:,:) = Z*t - reshape(M(i)*dMdi(j,:,:) + dMdi(i,:,:)*M(j),n,D);
    dSdi(j,i,:,:) = dSdi(i,j,:,:);
    dSdX(j,i,:,:) = dSdX(i,j,:,:);
  end % j

  S(i,i) = S(i,i) + 1e-06;    % add small diagonal jitter for numerical reasons
end % i

dSdX(:,:,D+1,:) = -dSdX(:,:,D+2,:);
dSdX(:,:,D+1,:) = -dSdX(:,:,D+2,:);

% 4) centralize moments
S = S - M*M';
%S(diag(S)<0,diag(S)<0) = 1e-6;

% 5) Vectorize derivatives
dMds=reshape(dMds,[E D*D]);
dSdm=reshape(dSdm,[E*E D]); dSds=reshape(dSds,[E*E D*D]);
dVdm=reshape(dVdm,[D*E D]); dVds=reshape(dVds,[D*E D*D]);
dMdi=reshape(dMdi,E,[]);  dMdt=reshape(dMdt,E,[]);  dMdX=reshape(dMdX,E,[]);
dSdi=reshape(dSdi,E*E,[]);dSdt=reshape(dSdt,E*E,[]);dSdX=reshape(dSdX,E*E,[]);
dVdi=reshape(dVdi,D*E,[]);dVdt=reshape(dVdt,D*E,[]);dVdX=reshape(dVdX,D*E,[]);
\end{lstlisting}

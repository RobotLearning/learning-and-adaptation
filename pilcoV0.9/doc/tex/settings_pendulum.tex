
% This LaTeX was auto-generated from an M-file by MATLAB.
% To make changes, update the M-file and republish this document.



    
    
      \subsection{settings\_pendulum.m}

\begin{par}
\textbf{Summary:} Script set up the pendulum scenario
\end{par} \vspace{1em}
\begin{par}
Copyright (C) 2008-2013 by Marc Deisenroth, Andrew McHutchon, Joe Hall, and Carl Edward Rasmussen.
\end{par} \vspace{1em}
\begin{par}
Last modified: 2013-05-24
\end{par} \vspace{1em}


\subsection*{High-Level Steps} 

\begin{enumerate}
\setlength{\itemsep}{-1ex}
   \item Define state and important indices
   \item Set up scenario
   \item Set up the plant structure
   \item Set up the policy structure
   \item Set up the cost structure
   \item Set up the GP dynamics model structure
   \item Parameters for policy optimization
   \item Plotting verbosity
   \item Some array initializations
\end{enumerate}


\subsection*{Code} 


\begin{lstlisting}
warning('off','all'); format short; format compact;

% include some paths
try
  rd = '../../';
  addpath([rd 'base'],[rd 'util'],[rd 'gp'],[rd 'control'],[rd 'loss']);
catch
end

rand('seed',5); randn('seed',13);



% 1. Define state and important indices

% 1a. Full state representation (including all augmentations)
%  1  dtheta        angular velocity of inner pendulum
%  2  theta2        angle inner pendulum
%  3  sin(theta)    complex representation ...
%  4  cos(theta)    ... of angle of pendulum
%  5  u             torque applied to the pendulum

% 1b. Important indices
% odei  indicies for the ode solver
% augi  indicies for variables augmented to the ode variables
% dyno  indicies for the output from the dynamics model and indicies to loss
% angi  indicies for variables treated as angles (using sin/cos representation)
% dyni  indicies for inputs to the dynamics model
% poli  indicies for variables that serve as inputs to the policy
% difi  indicies for training targets that are differences (rather than values)

odei = [1 2];               % varibles for the ODE solver
augi = [];                  % variables to be augmented
dyno = [1 2];               % variables to be predicted (and known to loss)
angi = [2];                 % angle variables
dyni = [1 3 4];             % variables that serve as inputs to the dynamics GP
poli = [1 3 4];             % variables that serve as inputs to the policy
difi = [1 2];               % variables that are learned via differences

% 2. Set up the scenario
dt = 0.1;                      % [s] sampling time
T = 4;                         % [s] prediction time
H = ceil(T/dt);                % prediction steps (optimization horizon)
mu0 = [0 0]';                  % initial state mean
S0 = 0.01*eye(2);              % initial state variance
N = 10;                        % number of policy optimizations
J = 1;                         % no. of inital training rollouts (of length H)
K = 1;                         % number of initial states for which we optimize
nc = 20;                       % size of controller training set

% 3. Set up the plant structure
plant.dynamics = @dynamics_pendulum;    % dynamics ODE function
plant.noise = diag([0.1^2 0.01^2]);     % measurement noise
plant.dt = dt;
plant.ctrl = @zoh;                  % controler is zero-order-hold
plant.odei = odei;                  % indices of the varibles for the ODE solver
plant.augi = augi;                  % indices of augmented variables
plant.angi = angi;
plant.poli = poli;
plant.dyno = dyno;
plant.dyni = dyni;
plant.difi = difi;
plant.prop = @propagated;

% 4. Set up the policy structure
policy.fcn = @(policy,m,s)conCat(@congp,@gSat,policy,m,s);% controller
                                                          % representation
policy.maxU = 2.5;                                        % max. amplitude of
                                                          % torque
[mm ss cc] = gTrig(mu0, S0, plant.angi);                  % represent angles
mm = [mu0; mm]; cc = S0*cc; ss = [S0 cc; cc' ss];         % in complex plane
policy.p.inputs = gaussian(mm(poli), ss(poli,poli), nc)'; % init. location of
                                                          % basis functions
policy.p.targets = 0.1*randn(nc, length(policy.maxU));    % init. policy targets
                                                          % (close to zero)
policy.p.hyp = log([1 0.7 0.7 1 0.01]');                  % initialize
                                                          % hyper-parameters



% 5. Set up the cost structure
cost.fcn = @loss_pendulum;                           % cost function
cost.gamma = 1;                                      % discount factor
cost.p = 1.0;                                        % length of pendulum
cost.width = 0.5;                                    % cost function width
cost.expl = 0;                                       % exploration parameter
cost.angle = plant.angi;                             % angle variables in cost
cost.target = [0 pi]';                               % target state


% 6. Set up the GP dynamics model structure
dynmodel.fcn = @gp1d;                % function for GP predictions
dynmodel.train = @train;             % function to train dynamics model
dynmodel.induce = zeros(300,0,1);    % shared inducing inputs (sparse GP)
trainOpt = [300 500];                % defines the max. number of line searches
                                     % when training the GP dynamics models
                                     % trainOpt(1): full GP,
                                     % trainOpt(2): sparse GP (FITC)

% 7. Parameters for policy optimization
opt.length = 75;                         % max. number of line searches
opt.MFEPLS = 30;                         % max. number of function evaluations
                                         % per line search
opt.verbosity = 1;                       % verbosity: specifies how much
                                         % information is displayed during
                                         % policy learning. Options: 0-3
opt.method = 'BFGS';                     % optimization algorithm. Options:
                                         % 'BFGS' (default), 'LBFGS', 'CG'


% 8. Plotting verbosity
plotting.verbosity = 1;            % 0: no plots
                                   % 1: some plots
                                   % 2: all plots

% 9. Some initializations
x = []; y = [];
fantasy.mean = cell(1,N); fantasy.std = cell(1,N);
realCost = cell(1,N); M = cell(N,1); Sigma = cell(N,1);
\end{lstlisting}


% This LaTeX was auto-generated from an M-file by MATLAB.
% To make changes, update the M-file and republish this document.



    
    
      \subsection{concat.m}

\begin{par}
\textbf{Summary:} Compute a control signal $u$ from a state distribution $x\sim\mathcal N(x|m,s)$. Here, the predicted control distribution and its derivatives are computed by concatenating a controller "con" with a saturation function "sat", such as gSat.m.
\end{par} \vspace{1em}

\begin{verbatim}function [M, S, C, dMdm, dSdm, dCdm, dMds, dSds, dCds, dMdp, dSdp, dCdp] ...
         = conCat(con, sat, policy, m, s)\end{verbatim}
    
\begin{verbatim}Example call: conCat(@congp, @gSat, policy, m, s)\end{verbatim}
    \begin{par}
\textbf{Input arguments:}
\end{par} \vspace{1em}
\begin{verbatim}con       function handle (controller)
sat       function handle (squashing function)
policy    policy structure
  .maxU   maximum amplitude of control signal (after squashing)
m         mean of input distribution                             [D x 1]
s         covariance of input distribution                       [D x D]\end{verbatim}
\begin{par}
\textbf{Output arguments:}
\end{par} \vspace{1em}
\begin{verbatim}M         control mean                                           [E   x   1]
S         control covariance                                     [E   x   E]
C         inv(s)*cov(x,u)                                        [D   x   E]
dMdm      deriv. of expected control wrt input mean              [E   x   D]
dSdm      deriv. of control covariance wrt input mean            [E*E x   D]
dCdm      deriv. of C wrt input mean                             [D*E x   D]
dMds      deriv. of expected control wrt input covariance        [E   x D*D]
dSds      deriv. of control covariance wrt input covariance      [E*E x D*D]
dCds      deriv. of C wrt input covariance                       [D*E x D*D]
dMdp      deriv. of expected control wrt policy parameters       [E   x   P]
dSdp      deriv. of control covariance wrt policy parameters     [E*E x   P]
dCdp      deriv. of C wrt policy parameters                      [D*E x   P]\end{verbatim}
\begin{verbatim}where P is the total number of policy parameters\end{verbatim}
\begin{par}
Copyright (C) 2008-2013 by Marc Deisenroth, Andrew McHutchon, Joe Hall, and Carl Edward Rasmussen.
\end{par} \vspace{1em}
\begin{par}
Last modified: 2012-07-03
\end{par} \vspace{1em}


\subsection*{High-Level Steps} 

\begin{enumerate}
\setlength{\itemsep}{-1ex}
   \item Compute unsquashed control signal
   \item Compute squashed control signal
\end{enumerate}

\begin{lstlisting}
function [M, S, C, dMdm, dSdm, dCdm, dMds, dSds, dCds,  dMdp, dSdp, dCdp] ...
  = conCat(con, sat, policy, m, s)
\end{lstlisting}


\subsection*{Code} 


\begin{lstlisting}
maxU=policy.maxU; % amplitude limit of control signal
E=length(maxU);   % dimension of control signal
D=length(m);      % dimension of input

% pre-compute some indices
F=D+E; j=D+1:F; i=1:D;
% initialize M and S
M = zeros(F,1); M(i) = m; S = zeros(F); S(i,i) = s;

if nargout < 4   % without derivatives
  [M(j), S(j,j), Q] = con(policy, m, s);  % compute unsquashed control signal v
  q = S(i,i)*Q; S(i,j) = q; S(j,i) = q';  % compute joint covariance S=cov(x,v)
  [M, S, R] = sat(M, S, j, maxU);         % compute squashed control signal u
  C = [eye(D) Q]*R;                       % inv(s)*cov(x,u)
else             % with derivatives
  Mdm = zeros(F,D); Sdm = zeros(F*F,D); Mdm(1:D,1:D) = eye(D);
  Mds = zeros(F,D*D); Sds = kron(Mdm,Mdm);

  X = reshape(1:F*F,[F F]); XT = X';                  % vectorized indices
  I=0*X;I(j,j)=1; jj=X(I==1)'; I=0*X;I(i,j)=1;ij=X(I==1)'; ji=XT(I==1)';

  % 1. Unsquashed controller --------------------------------------------------
  [M(j), S(j,j), Q, Mdm(j,:), Sdm(jj,:), dQdm, Mds(j,:), ...
    Sds(jj,:), dQds, Mdp, Sdp, dQdp] = con(policy, m, s);
  q = S(i,i)*Q; S(i,j) = q; S(j,i) = q';  % compute joint covariance S=cov(x,v)

  % update the derivatives
  SS = kron(eye(E),S(i,i)); QQ = kron(Q',eye(D));
  Sdm(ij,:) = SS*dQdm;      Sdm(ji,:) = Sdm(ij,:);
  Sds(ij,:) = SS*dQds + QQ; Sds(ji,:) = Sds(ij,:);

  % 2. Apply Saturation -------------------------------------------------------
  [M, S, R, MdM, SdM, RdM, MdS, SdS, RdS] = sat(M, S, j, maxU);

  % apply chain-rule to compute derivatives after concatenation
  dMdm = MdM*Mdm + MdS*Sdm; dMds = MdM*Mds + MdS*Sds;
  dSdm = SdM*Mdm + SdS*Sdm; dSds = SdM*Mds + SdS*Sds;
  dRdm = RdM*Mdm + RdS*Sdm; dRds = RdM*Mds + RdS*Sds;

  dMdp = MdM(:,j)*Mdp + MdS(:,jj)*Sdp;
  dSdp = SdM(:,j)*Mdp + SdS(:,jj)*Sdp;
  dRdp = RdM(:,j)*Mdp + RdS(:,jj)*Sdp;

  C = [eye(D) Q]*R; % inv(s)*cov(x,u)
  % update the derivatives
  RR = kron(R(j,:)',eye(D)); QQ = kron(eye(E),[eye(D) Q]);
  dCdm = QQ*dRdm + RR*dQdm;
  dCds = QQ*dRds + RR*dQds;
  dCdp = QQ*dRdp + RR*dQdp;
end
\end{lstlisting}

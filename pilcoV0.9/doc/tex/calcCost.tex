
% This LaTeX was auto-generated from an M-file by MATLAB.
% To make changes, update the M-file and republish this document.



    
    
      \subsection{calcCost.m}

\begin{par}
\textbf{Summary:} Function to calculate the incurred cost and its standard deviation, given a sequence of predicted state distributions and the cost struct
\end{par} \vspace{1em}
\begin{verbatim}[L sL] = calcCost(cost, M, S)\end{verbatim}
\begin{par}
\textbf{Input arguments:}
\end{par} \vspace{1em}
\begin{verbatim}cost               cost structure
M                  mean vectors of state trajectory (D-by-H matrix)
S                  covariance matrices at each time step (D-by-D-by-H)\end{verbatim}
\begin{par}
\textbf{Output arguments:}
\end{par} \vspace{1em}
\begin{verbatim}L                  expected incurred cost of state trajectory
sL                 standard deviation of incurred cost\end{verbatim}
\begin{par}
Copyright (C) 2008-2013 by Marc Deisenroth, Andrew McHutchon, Joe Hall, and Carl Edward Rasmussen.
\end{par} \vspace{1em}
\begin{par}
Last modified: 2013-01-23
\end{par} \vspace{1em}


\subsection*{High-Level Steps} 

\begin{enumerate}
\setlength{\itemsep}{-1ex}
   \item Augment state distribution with trigonometric functions
   \item Compute distribution of the control signal
   \item Compute dynamics-GP prediction
   \item Compute distribution of the next state
\end{enumerate}

\begin{lstlisting}
function [L sL] = calcCost(cost, M, S)
\end{lstlisting}


\subsection*{Code} 


\begin{lstlisting}
H = size(M,2);                                             % horizon length
L = zeros(1,H); SL = zeros(1,H);

% for each time step, compute the expected cost and its variance
for h = 1:H
  [L(h),d1,d2,SL(h)]  = cost.fcn(cost, M(:,h), S(:,:,h));
end

sL = sqrt(SL);                                         % standard deviation
\end{lstlisting}

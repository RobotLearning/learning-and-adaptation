
% This LaTeX was auto-generated from an M-file by MATLAB.
% To make changes, update the M-file and republish this document.



    
    
      \subsection{propagate.m}

\begin{par}
\textbf{Summary:} Propagate the state distribution one time step forward.
\end{par} \vspace{1em}

\begin{verbatim}[Mnext, Snext] = propagate(m, s, plant, dynmodel, policy)\end{verbatim}
    \begin{par}
\textbf{Input arguments:}
\end{par} \vspace{1em}
\begin{verbatim}m                 mean of the state distribution at time t           [D x 1]
s                 covariance of the state distribution at time t     [D x D]
plant             plant structure
dynmodel          dynamics model structure
policy            policy structure\end{verbatim}
\begin{par}
\textbf{Output arguments:}
\end{par} \vspace{1em}
\begin{verbatim}Mnext             mean of the successor state at time t+1            [E x 1]
Snext             covariance of the successor state at time t+1      [E x E]\end{verbatim}
\begin{par}
Copyright (C) 2008-2013 by Marc Deisenroth, Andrew McHutchon, Joe Hall, Henrik Ohlsson, and Carl Edward Rasmussen.
\end{par} \vspace{1em}
\begin{par}
Last modified: 2013-01-23
\end{par} \vspace{1em}


\subsection*{High-Level Steps} 

\begin{enumerate}
\setlength{\itemsep}{-1ex}
   \item Augment state distribution with trigonometric functions
   \item Compute distribution of the control signal
   \item Compute dynamics-GP prediction
   \item Compute distribution of the next state
\end{enumerate}

\begin{lstlisting}
function [Mnext, Snext] = propagate(m, s, plant, dynmodel, policy)
\end{lstlisting}


\subsection*{Code} 


\begin{lstlisting}
% extract important indices from structures
angi = plant.angi;  % angular indices
poli = plant.poli;  % policy indices
dyni = plant.dyni;  % dynamics-model indices
difi = plant.difi;  % state indices where the model was trained on differences

D0 = length(m);                                        % size of the input mean
D1 = D0 + 2*length(angi);          % length after mapping all angles to sin/cos
D2 = D1 + length(policy.maxU);          % length after computing control signal
D3 = D2 + D0;                                         % length after predicting
M = zeros(D3,1); M(1:D0) = m; S = zeros(D3); S(1:D0,1:D0) = s;   % init M and S

% 1) Augment state distribution with trigonometric functions ------------------
i = 1:D0; j = 1:D0; k = D0+1:D1;
[M(k), S(k,k) C] = gTrig(M(i), S(i,i), angi);
q = S(j,i)*C; S(j,k) = q; S(k,j) = q';

mm=zeros(D1,1); mm(i)=M(i); ss(i,i)=S(i,i)+diag(exp(2*dynmodel.hyp(end,:))/2);
[mm(k), ss(k,k) C] = gTrig(mm(i), ss(i,i), angi);     % noisy state measurement
q = ss(j,i)*C; ss(j,k) = q; ss(k,j) = q';

% 2) Compute distribution of the control signal -------------------------------
i = poli; j = 1:D1; k = D1+1:D2;
[M(k) S(k,k) C] = policy.fcn(policy, mm(i), ss(i,i));
q = S(j,i)*C; S(j,k) = q; S(k,j) = q';

% 3) Compute dynamics-GP prediction              ------------------------------
ii = [dyni D1+1:D2]; j = 1:D2;
if isfield(dynmodel,'sub'), Nf = length(dynmodel.sub); else Nf = 1; end
for n=1:Nf                               % potentially multiple dynamics models
  [dyn i k] = sliceModel(dynmodel,n,ii,D1,D2,D3); j = setdiff(j,k);
  [M(k), S(k,k), C] = dyn.fcn(dyn, M(i), S(i,i));
  q = S(j,i)*C; S(j,k) = q; S(k,j) = q';

  j = [j k];                                   % update 'previous' state vector
end

% 4) Compute distribution of the next state -----------------------------------
P = [zeros(D0,D2) eye(D0)]; P(difi,difi) = eye(length(difi));
Mnext = P*M; Snext = P*S*P'; Snext = (Snext+Snext')/2;
\end{lstlisting}

\begin{lstlisting}
function [dyn i k] = sliceModel(dynmodel,n,ii,D1,D2,D3) % separate sub-dynamics
% A1) Separate multiple dynamics models ---------------------------------------
if isfield(dynmodel,'sub')
  dyn = dynmodel.sub{n}; do = dyn.dyno; D = length(ii)+D1-D2;
  if isfield(dyn,'dyni'), di=dyn.dyni; else di=[]; end
  if isfield(dyn,'dynu'), du=dyn.dynu; else du=[]; end
  if isfield(dyn,'dynj'), dj=dyn.dynj; else dj=[]; end
  i = [ii(di) D1+du D2+dj]; k = D2+do;
  dyn.inputs = [dynmodel.inputs(:,[di D+du]) dynmodel.target(:,dj)];   % inputs
  dyn.target = dynmodel.target(:,do);                                 % targets
else
  dyn = dynmodel; k = D2+1:D3; i = ii;
end
\end{lstlisting}


% This LaTeX was auto-generated from an M-file by MATLAB.
% To make changes, update the M-file and republish this document.



    
    

\subsection*{rewrap.m} 

\begin{par}
\textbf{Summary:} Map the numerical elements in the vector $v$ onto the variables $s$, which can be of any type. The number of numerical elements must match; on exit, $v$ should be empty. Non-numerical entries are just copied. See also the reverse unwrap.m.
\end{par} \vspace{1em}

\begin{verbatim}  [s v] = rewrap(s, v)\end{verbatim}
    \begin{par}
\textbf{Input arguments:}
\end{par} \vspace{1em}
\begin{verbatim}s     structure, cell, or numeric values
v     structure, cell, or numeric values\end{verbatim}
\begin{par}
\textbf{Output arguments:}
\end{par} \vspace{1em}
\begin{verbatim}s     structure, cell, or numeric values
v     [empty]\end{verbatim}
\begin{par}
Copyright (C) 2008-2013 by Marc Deisenroth, Andrew McHutchon, Joe Hall, and Carl Edward Rasmussen.
\end{par} \vspace{1em}
\begin{par}
Last modified: 2013-03-25
\end{par} \vspace{1em}

\begin{lstlisting}
function [s v] = rewrap(s, v)
\end{lstlisting}


\subsection*{Code} 


\begin{lstlisting}
if isnumeric(s)
  if numel(v) < numel(s)
    error('The vector for conversion contains too few elements')
  end
  s = reshape(v(1:numel(s)), size(s));            % numeric values are reshaped
  v = v(numel(s)+1:end);                        % remaining arguments passed on
elseif isstruct(s)
  [s p] = orderfields(s); p(p) = 1:numel(p);      % alphabetize, store ordering
  [t v] = rewrap(struct2cell(s), v);                 % convert to cell, recurse
  s = orderfields(cell2struct(t,fieldnames(s),1),p);  % conv to struct, reorder
elseif iscell(s)
  for i = 1:numel(s)             % cell array elements are handled sequentially
    [s{i} v] = rewrap(s{i}, v);
  end
end
\end{lstlisting}

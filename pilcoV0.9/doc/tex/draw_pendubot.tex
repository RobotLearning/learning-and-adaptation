
% This LaTeX was auto-generated from an M-file by MATLAB.
% To make changes, update the M-file and republish this document.



    
    

\subsection*{draw\_pendubot.m} 

\begin{par}
\textbf{Summary:} Draw the Pendubot system with reward, applied torque, and predictive uncertainty of the tips of the pendulums
\end{par} \vspace{1em}

\begin{verbatim}  function draw_pendubot(theta1, theta2, force, cost, text1, text2, M, S)\end{verbatim}
    \begin{par}
\textbf{Input arguments:}
\end{par} \vspace{1em}
\begin{verbatim}theta1     angle of inner pendulum
theta2     angle of outer pendulum
f1         torque applied to inner pendulum
f2         torque applied to outer pendulum
cost       cost structure
  .fcn     function handle (it is assumed to use saturating cost)
  .\ensuremath{<}\ensuremath{>}      other fields that are passed to cost
text1      (optional) text field 1
text2      (optional) text field 2
M          (optional) mean of state
S          (optional) covariance of state\end{verbatim}
\begin{par}
Copyright (C) 2008-2013 by Marc Deisenroth, Andrew McHutchon, Joe Hall, and Carl Edward Rasmussen.
\end{par} \vspace{1em}
\begin{par}
Last modified: 2013-03-08
\end{par} \vspace{1em}

\begin{lstlisting}
function draw_pendubot(theta1, theta2, force, cost, text1, text2, M, S)
\end{lstlisting}


\subsection*{Code} 


\begin{lstlisting}
l = 0.6;
xmin = -2*l;
xmax = 2*l;
umax = 2;
height = 0;

% Draw double pendulum
clf; hold on
sth1 = sin(theta1); sth2 = sin(theta2);
cth1 = cos(theta1); cth2 = cos(theta2);
pendulum1 = [0, 0; -l*sth1, l*cth1];
pendulum2 = [-l*sth1, l*cth1; -l*(sth1-sth2), l*(cth1+cth2)];
plot(pendulum1(:,1), pendulum1(:,2),'r','linewidth',4)
plot(pendulum2(:,1), pendulum2(:,2),'r','linewidth',4)

% plot target location
plot(0,2*l,'k+','MarkerSize',20);
plot([xmin, xmax], [-height, -height],'k','linewidth',2)
% plot inner joint
plot(0,0,'k.','markersize',24)
plot(0,0,'y.','markersize',14)
% plot outer joint
plot(-l*sth1, l*cth1,'k.','markersize',24)
plot(-l*sth1, l*cth1,'y.','markersize',14)
% plot tip of outer joint
plot(-l*(sth1-sth2), l*(cth1+cth2),'k.','markersize',24)
plot(-l*(sth1-sth2), l*(cth1+cth2),'y.','markersize',14)
plot(0,-2*l,'.w','markersize',0.005)

% % Draw sample positions of the joints
% if nargin > 6
%   samples = gaussian(M,S+1e-8*eye(4),1000);
%   t1 = samples(3,:); t2 = samples(4,:);
%   plot(-l*sin(t1),l*cos(t1),'b.','markersize',2)
%   plot(-l*(sin(t1)-sin(t2)),l*(cos(t1)+cos(t2)),'r.','markersize',2)
% end

% plot ellipses around tips of pendulums (if M, S exist)
try
  if max(max(S))>0
    [M1 S1 M2 S2] = getPlotDistr_pendubot(M, S, l, l);
    error_ellipse(S1, M1, 'style','b'); % inner pendulum
    error_ellipse(S2, M2, 'style','r'); % outer pendulum
  end
catch
end

% Draw useful information
% plot applied torque
plot([0 force/umax*xmax],[-0.5, -0.5],'g','linewidth',10)
% plot immediate reward
reward = 1-cost.fcn(cost,[0, 0, theta1, theta2]',zeros(4));
plot([0 reward*xmax],[-0.7, -0.7],'y','linewidth',10)
text(0,-0.5,'applied  torque (inner joint)')
text(0,-0.7,'immediate reward')
if exist('text1','var')
  text(0,-0.9, text1)
end
if exist('text2','var')
  text(0,-1.1, text2)
end

set(gca,'DataAspectRatio',[1 1 1],'XLim',[xmin xmax],'YLim',[-2*l 2*l]);
axis off
drawnow;
\end{lstlisting}
